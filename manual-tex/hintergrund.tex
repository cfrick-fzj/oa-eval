\section{Hintergrund}
\label{hintergrund}
In Vorbereitung auf die Antragstellung bei der Deutschen Forschungsgemeinschaft (DFG) zur Förderung eines Open-Access-Publikationsfonds wurde an der Technischen Universität Berlin (TU Berlin) folgendes Verfahren für die Analyse des Publikationsaufkommens entwickelt. Dieses Verfahren wurde weiterentwickelt für die Analyse des Open-Access-Anteils wissenschaftlicher Zeitschriftenartikel der Angehörigen von Berliner Bildungs- und Forschungseinrichtungen.

Für die DFG-Antragsstellung benötigt wurden Aussagen über das Aufkommen von Zeitschriftenartikeln von TU-Angehörigen für die  Jahre 2014 und 2015, insbesondere über den Anteil von Artikeln in Open-Access-Zeitschriften. Für die Analyse des Berliner Publikationsaufkommens von neun verschiedenen Einrichtungen wurden die Jahre 2013 bis 2015 analysiert. 

Für die Analyse wurde auf Daten aus zehn bzw. sechzehn externen Literatur- und Zitationsdatenbanken zurückgegriffen.
Die gewonnenen Daten zu den Dokumententypen \texttt{Article} bzw. \texttt{Review} wurden normalisiert, aggregiert und auf Dubletten geprüft. Um Artikel aus Open-Access-Zeitschriften zu identifizieren, wurden Daten des \textit{Directory of Open Access Journals (DOAJ)}\footnote{Directory of Open Access Journals (DOAJ) s. \url{http://doaj.org}} genutzt. In den verbleibenden Daten von Open-Access-Artikeln wurden im Folgenden diejenigen Artikel identifiziert, für die Angehörige der untersuchten Einrichtungen als Erst- oder Korrespondenzautoren angegeben sind.

Bei der Datenerhebung wurden folgende Datenbanken berücksichtigt: \textit{Web of Science Core Collection}, \textit{SciFinder (CAPlus)}, \textit{PubMed}, \textit{TEMA}, \textit{Inspec}, \textit{IEEE Xplore}, \textit{ProQuest Social Sciences}\footnote{Das Datenbankpaket \textit{Social Sciences} besteht aus folgenden Einzeldatenbanken: Applied Social Sciences Index and Abstracts (ASSIA), British Periodicals, Digital National Security Archive, ebrary e-books, ERIC, Index Islamicus, International Bibliography of the Social Sciences (IBSS), PAIS International, Periodicals Archive Online, Periodicals Index Online, Physical Education Index, PILOTS: Published International Literature On Traumatic Stress, ProQuest Dissertations \& Theses Full Text: The Humanities and Social Sciences Collection, ProQuest Dissertations \& Theses Global: Social Sciences, Social Services Abstracts, Sociological Abstracts, Worldwide Political Science Abstracts}, \textit{Business Source Complete}, \textit{GeoRef}, \textit{CABAbstracts}, \textit{CINAHL}, \textit{Academic Search Premier}, \textit{Embase}, \textit{LISA}, \textit{Scopus}, \textit{SportDiscus}.

Zur Unterstützung der Evaluation wurde ein Python-Skript entwickelt, dessen Funktionsweise ab S.~\pageref{funktionsweise} beschrieben wird. Ab S.~\pageref{analysis} wird erläutert, welche Schritte für eine eigene Analyse mithilfe dieses Skripts durchzuführen sind.

Ob mit dem hier beschriebenen Verfahren alle Open-Access-Artikel einer bestimmten Einrichtung identifiziert werden können, bleibt offen. Folgende Faktoren stellen potentielle Fehlerquellen dar:
\begin{itemize}
\item Artikel in Open-Access-Zeitschriften, die nicht in einer der geprüften Datenbanken indexiert sind, werden nicht berücksichtigt.
\item Die Artikel werden über externe Datenbanken ermittelt; Voraussetzung für die Identifizierung ist das Erfassen der Affiliation in diesen Datenbanken. Es werden hier zwar Affiliationen für alle Autorinnen und Autoren erfasst -- allerdings pro Autor bzw. Autorin meist nur eine Affiliation. Bei Mehrfachaffiliationen wird i.\,d.\,R. nur eine Institution in der Datenbank erfasst.
\item Open-Access-Zeitschriften werden mithilfe des \textit{DOAJ} identifiziert. Ist eine Zeitschrift nicht im \textit{DOAJ} erfasst, werden Open-Access-Artikel nicht als solche erkannt. Es wird zudem nur berücksichtigt, ob die Zeitschrift zum Zeitpunkt der Analyse im \textit{DOAJ} verzeichnet ist. Es ist also denkbar, dass die Zeitschrift zum Zeitpunkt der Publikation des Artikels noch unter einem Closed-Access-Modell operierte. Es ist ebenso möglich, dass die Zeitschrift zum Zeitpunkt der Publikation des Artikels noch als Open-Access-Zeitschrift im \textit{DOAJ} gelistet wurde, zum Zeitpunkt der Analyse aber aus dem \textit{DOAJ} entfernt wurde.
\item Während im \textit{Web of Science (WoS)} die Korrespondenzautorin bzw. der Korrespondenzautor als \textit{reprint author} gesondert ausgewiesen wird, werden in anderen Datenbanken lediglich Affiliationen erfasst. Ein eindeutiger Rückschluss auf die Korrespondenzautorschaft ist für diese Daten nicht möglich. Es werden für diese Datenbanken daher lediglich die Institutionsangaben zu Erstautorinnen bzw. Erstautoren evaluiert. Nicht in allen Disziplinen aber sind Korrespondenz- und Erstautorin identisch. Open-Access-Artikel, für die Angehörige einer Einrichtung Korrespondenz- nicht aber Erstautoren sind, bleiben somit unentdeckt.
\end{itemize}